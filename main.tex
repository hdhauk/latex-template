% This is a document written to introduce students in MATH 2300-04 at FSU to LaTeX and Overleaf.  Any other students are free to use this as well.

% All of this stuff with '%' in front is a comment and ignored by the compiler.
%
% The lines before the "\begin{document}" line is called the preamble.
% This is where you load particular packages you need.
% Until you are more experienced, or the program says you are missing packages, it is safe to ignore it.
%
%----------------------------------

\documentclass[12pt]{article}
\usepackage[margin=1in]{geometry}% Change the margins here if you wish.
\setlength{\parindent}{0pt} % This is the set the indent length for new paragraphs, change if you want.
\setlength{\parskip}{5pt} % This sets the distance between paragraphs, which will be used anytime you have a blank line in your LaTeX code.
\pagenumbering{gobble}% This means the page will not be numbered. You can comment it out if you like page numbers.


%These packages allow the most of the common "mathly things"
\usepackage{amsmath,amsthm,amssymb}

%This package allows you to add graphs or any other images.
\usepackage{graphicx}

%These are the packages I usually use and needed for this document. There are bajillions of others to do nearly antyhing you want.
\usepackage{color}
\usepackage{enumerate}
\usepackage{multicol}
\usepackage[utf8]{inputenc}
\usepackage[T1]{fontenc}|12v, 
\usepackage{lmodern}
\usepackage{amsfonts}

\begin{document}

\begin{titlepage}
  \centering
	\includegraphics[width=7cm]{ntnu-logo}\par\vspace{1cm}
	{\scshape\LARGE Norwegian University \\ {\Large of} \\ Science and Technology \par}
	\vspace{1cm}
	{\scshape\Large [COURSE CODE] \par}
	\vspace{1.5cm}
	{\huge\bfseries [Title]\par}
	\vspace{2cm}
	{\Large [NAME]\par}
	\vfill

	\vfill

% Bottom of the page
	{\large \today\par}
\end{titlepage}


\end{document}
