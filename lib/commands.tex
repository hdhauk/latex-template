% - Commands defined in this file:                                   - %
% -- \at{x} creates an "evaluated at x"-symbol: |_x                 -- %
% -- \avg{x} creates an "average of x"-symbol (bracket or overline) -- %
% -- \ceil{x} creates the ceiling of x                              -- %
% -- \cstitle creates a subsection with a centered title            -- %
% -- \ctitle creates a section with a centered title                -- %
% -- \floor{x} creates the floor of x                               -- %
% -- \grandtitle{title} creates a title which spans the entire page -- %
% -- \i{text} emphasizes the text and creates an index entry for it -- %
% -- \pdif{u}{x}{v} creates a complete partial derivative (du/dx)_v -- %
% -- \spdif{f}{x} creates a simple partial derivative: df/dx        -- %
% -- \wip creates a footnote with a message defined in the config   -- %

% Evaluated-at operator
\newcommand{\at}[1]{\Bigr|_{#1}}
% Average operator
\newcommand{\avg}[1]{\ifbool{bracketavg}{\left< #1 \right>}{\overline{#1}}}
% Mark unfinished text with a footnote
\newcommand{\wip}{\footnote{\wipmessage}}
% Centered section titles and subsection titles
\newcommand{\ctitle}[1]
{{
	\color{\maincolor}
	% \begin{centering}
    \section{#1}
    % \end{centering}
	\hrule height 0.5pt \bigskip
}}
\newcommand{\cstitle}[1]
{{
	\color{\maincolor}
	\begin{centering}\subsection{#1}\end{centering}
}}
\newcommand{\csection}[1]
{{
    \color{\maincolor}
    \reversemarginpar
    \marginnote{\subsection{#1}}[-1.35em]
}}




% Indexing: \i{text} emphasizes the content and creates an index entry for it
\ifbool{index}{\makeindex\renewcommand{\i}[1]{\index{#1}\emph{#1}}}{}
% 
\ifbool{chapnum}{\numberwithin{equation}{section}}{}
% Same display style for all math
\ifbool{dispstyle}{\everymath{\displaystyle}}{}
% Quotes
\ifbool{quotes}{
				\setlength\epigraphwidth{17cm}
				\setlength\epigraphrule{0pt}
				}{}
% Vector notation
%\renewcommand{\vec}[1]{\ifbool{arrowvec}{\overrightarrow{#1}}{\mathbf{#1}}}
\ifbool{boldvec}{\renewcommand{\vec}[1]{\bm{#1}}}{}
% Make all equation numbers have equal size
\makeatletter\renewcommand{\maketag@@@}[1]{\hbox{\m@th\ifbool{smalleqnum}{\small}{}\normalfont#1}}\makeatother
% Simple partial derivative
\newcommand{\spdif}[2]{\frac{\partial #1}{\partial #2}}
\newcommand{\tspdif}[2]{\tfrac{\partial #1}{\partial #2}}
% Partial derivative with #3 held constant
\newcommand{\pdif}[3]{\left(\frac{\partial #1}{\partial #2}\right)_{#3}}
% Rounding
\newcommand{\ceil}[1]{\left\lceil{#1}\right\rceil}
\newcommand{\floor}[1]{\left\lfloor{#1}\right\rfloor}
% Grandiose document title (requires graphicx)
\newcommand{\grandtitle}[1]
{{
	\color{\maincolor}
	\begin{centering}
	\resizebox{1\hsize}{!}
	{
		#1
	}
	\bigskip
	\end{centering}
}}

% Quotes
\makeatletter
				\patchcmd{\epigraph}{\@epitext{#1}}{\itshape\@epitext{#1}}{}{}
				\makeatother


\hypersetup{
    %bookmarks=true,         % show bookmarks bar?
    bookmarksnumbered=true,
    unicode=false,          % non-Latin characters in Acrobat’s bookmarks
    pdftoolbar=true,        % show Acrobat’s toolbar?
    pdfmenubar=true,        % show Acrobat’s menu?
    pdfpagemode=UseOutlines,
    pdffitwindow=false,     % window fit to page when opened
    pdfstartview={FitW},    % fits the width of the page to the window
    %pdftitle={},    % title
    pdfauthor={Erlend},     % author
    %pdfsubject={Subject},   % subject of the document
    %pdfcreator={Creator},   % creator of the document
    %pdfproducer={Producer}, % producer of the document
    %pdfkeywords={keyword1, key2, key3}, % list of keywords
    pdfnewwindow=true,      % links in new PDF window
    colorlinks=true,       % false: boxed links; true: colored links
    linkcolor=black,          % color of internal links (change box color with linkbordercolor)
    linkbordercolor = black,
    citecolor=green,        % color of links to bibliography
    filecolor=magenta,      % color of file links
    urlcolor=cyan           % color of external links
}