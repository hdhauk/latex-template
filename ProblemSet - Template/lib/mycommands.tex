% Laplace operator
\newsavebox\foobox
\newlength{\foodim}
\newcommand{\slantbox}[2][0]{\mbox{%
        \sbox{\foobox}{#2}%
        \foodim=#1\wd\foobox
        \hskip \wd\foobox
        \hskip -0.5\foodim
        \pdfsave
        \pdfsetmatrix{1 0 #1 1}%
        \llap{\usebox{\foobox}}%
        \pdfrestore
        \hskip 0.5\foodim
}}
\def\Laplace{\slantbox[-.45]{$\mathscr{L}$}}


%Theorems
\theoremstyle{definition}
\newtheorem{defn}{\color{WildStrawberry}{Definisjon}}
\newtheorem{exmp}{\color{WildStrawberry}{Example}}
\theoremstyle{plain}
\newtheorem{thm}{\color{Lavender}{Teorem}}
\theoremstyle{remark}
\newtheorem*{egen}{\color{RoyalBlue}{Egenskaper}}
\newtheorem*{merk}{\color{Magenta}{Merk}}
\newtheorem{eks}{\color{WildStrawberry}{Eksempel}}


\newcommand{\e}[1]{\operatorname{e}^{#1}}
\newcommand{\V}{\mathcal{V}}
\newcommand{\interior}[1]{%
  {\kern0pt#1}^{\mathrm{o}}%
}
\DeclarePairedDelimiterX{\inp}[2]{\langle}{\rangle}{#1, #2}

% Streachable parenheses
\newcommand{\pa}[1]{\left(#1\right)} % encloses the argument using stretchable parentheses
% Streachable brackets
\newcommand{\bra}[1]{\left[#1\right]}
% Streachable curly brackets
\newcommand{\cur}[1]{\left\{#1 \right\}}
% Matrix without brackets
\newcommand{\mat}[1]{\begin{matrix}#1\end{matrix}}
% Matrix with brackets
\newcommand{\bmat}[1]{\bra{\mat{#1}}}
% Determinant matrix
\newcommand{\detmat}[1]{\begin{vmatrix}#1\end{vmatrix}}
% Numbering in {align*} enviroment. Usage: \numberthis \label{eq:MY_LABEL}
\newcommand{\numberthis}{\addtocounter{equation}{1}\tag{\theequation}}
